\documentclass[a4paper,12pt]{article}

\usepackage[utf8]{inputenc}

\usepackage{xcolor}

\author{S\"oren Edvinsson, Erling Lundevaller, and G\"ran Brostr\"om}

\title{Prolonged life and structure of mortality from a cause of death
  perspective, Sweden 1900-2050}

\begin{document}

\maketitle

\section{Purpose and aims}

The main aim of the project is to analyze the development of old age
mortality during the 20th century until present-day Sweden, and to simulate
future development of survival in old age. This will be performed by
analyses of changes in cause of death patterns, and from a cohort
perspective. A basic question is if the Swedish population is approaching a
situation where future increases in survival at old age and consequently
increasing life expectancy are slowing down or even ending. The idea of a
maximum life expectancy imply that the present population is reaching a
biological limit to human life span. Hypothetically, this is considered to
be related to what has been called rectangularization of mortality,
i.e. that the dispersion of age at death is becoming more concentrated to
the supposed limit when premature and “avoidable” deaths have been
increasingly controlled, while the remaining deaths are the consequence of
senescence, the biological aging process that is not modifiable in the same
way. An alternative scenario is that the aging process can be delayed,
shifting mortality towards higher ages. This does not exclude the
possibility of a biological limit, rather that we still cannot identify any
possible ending of increasing life expectancy. 

The process of declining mortality will be analyzed from a cause-of-death
perspective, decomposing the process into different death causes but also
analyzing age structure of mortality within separate groups of death
causes, i.e. if they take place at higher ages and are becoming
compressed. We will also analyze the dispersion in the major cause-of-death
categories will improve the explanatory value of the research since the
distribution of death ages varies considerably. A possible
rectangularization may be related to changing impact of different causes of
death. We also consider changes related to educational, socio-economic or
regional mortality differences. A changing distribution of the population
when it comes to social position and/or residence can influence how death
ages are dispersed. 

Another aspect to consider is that present-day health and survival among
Swedes is related to previous conditions. Experiences of economic crises,
disease episodes and other living conditions and experiences influences the
complete life course. We therefore analyze distribution of old age
mortality not only from a period perspective but also according to birth
cohorts when applicable, i.e. in older birth cohorts where all or the
majority of deaths has taken place, and the possible effects of later birth
cohorts that have lived their lives in substantially different health
environments. (Can we relate this to historical events or circumstances?) 

In the final step, we will simulate the possible future development of old
age mortality, partly by the results from the historical study where we
take into consideration both what we know about population structure when
it comes to educational level, but where we also make assumptions about the
future development of socio-economic differences, regional mortality
patterns and the effect of the development within certain cause-of-death
groups. 

\section{Survey of the field}

During the last two centuries, life expectancy has increased enormously, in
Sweden as well as in other parts of the world. Most of the increase can be
attributed to by declining mortality in young ages, as described in Omran’s
(1971) epidemiological transition model, describing the development in
three stages. This led many to assume a slowing down and even the end of
increases in longevity when mankind approaches a possible biological limit
to maximum length of life. Throughout the last century, several suggestions
of highest possible life expectancy for populations have been formulated
but quickly been refuted by reality (Vaupel 1997?). Life expectancy has
continued to increase, but with the fundamental change that the additional
years during the last decades are the consequence of improved survival in
old age. Evidence for this continued increase in survival have been
presented through many different measures. The development of highest
national life expectancy has increased at a remarkably steady pace from the
1840s onwards, maximum age at death has increased in a fairly similar way,
the modal age of mortality has also increased steadily during this process
and the number and proportion of centenarians and supercentenarians has
increased (Wilmoth and Lundström 1996; Oeppen and Vaupel 2002; REF). The
increasing number of centenarians and their age being better documented
makes it possible to perform more thorough analyses of the oldest-old. A
recent study of Drefahl et al (2012) showed however no changes in Sweden in
this age group.
 
When the new patterns of continued increased survival became identified in
the 1980s, Olshansky and Ault (1986) suggested a fourth additional state to
Omran’s epidemiological transition model – the age of delayed aging. The
additional years could to a large extent be attributed to decrease in
mortality from cardiovascular diseases. The revolution of CVD mortality is
thus one important explanation to the longer lives, but will this process
continue or are we reaching a situation when the transition is slowing
down? This is still an unresolved question that has led to revisions of the
epidemiologic transition model. An essential prerequisite in several of the
transition models has been an assumed rectangularization of death age in
populations (Fries 1980 and 1984). When mortality is delayed and life
expectancy approaches the biological limit, then dispersion of age at death
diminishes and becomes concentrated to a shorter age interval - the
survival curve takes a rectangular form. It is apparent that the age range
of deaths has become more concentrated over time if we look at mortality in
all age groups. The results may however differ if we restrict the analysis
to old age mortality. Engelman et al (2010) for example found that
dispersion in old age has increased during the last decades and that ”…
survivors to older ages have become increasingly heterogeneous.” Declining
mortality in younger ages may change the selection of frail people in the
ageing population, thus increasing the variability in mortality among
elderly. The conclusions rely therefore strongly on how the measure is
defined. Wilmoth and Horiuchi 1999  

One important aspect of the epidemiologic transition is the changing
cause-of-death pattern. Mortality during the pre-transitional phase was
dominated by infectious diseases, usually occurring in epidemic cycles. The
highest death toll were taken in younger age groups. As mortality from
infectious diseases started to decline, a higher proportion of death took
place in older age group, mainly due to different chronic
diseases. Cardiovascular and cancer diseases became the dominant causes of
death. 

A related issue concerns the pattern of increasing death risks as people
age. Several researchers have identified an interesting paradox in
mortality along the life span. Above a certain age, mortality increases
steadily, following quite closely a Gompertz distribution. This increase
changes however in the oldest age group, it decelerates. For the very old,
the mortality increase is lower than for the ages below. A plausible
explanation is that populations 
are heterogeneous when it comes to frailty. There is always a selection
effect, eliminating the frail in a much higher speed. When reaching the
highest ages, only the strongest are still alive, leaving us with a
selection of highly strong and fit people and thus changing the rate of
increase in mortality. 

The type of studies discussed above involves several methodological
issues. As mentioned, there are several ways of identifying
rectangularization. Some have used the total population, including all age
groups, thus documenting substantial decreases in dispersion. Others have
suggested alternative approaches, for example focusing only on mortality
among the elderly. The different methods tell us different stories and
answer different questions. These are things that need to be elaborated in
the planned project. Similar challenges are related to studies of
deceleration of mortality.  

\section{Project description}

\subsection{Theory}

The basic theoretical background is the epidemiological transition model as
developed Omran. It posits a change from a mortality pattern dominated by
infectious diseases and deaths in young age groups (Age of Pestilence and
Famine) to a mortality pattern where people die in old age and from chronic
diseases (Age of Degenerative and Man-made diseases) with an intermittent
stage in between (Age of Receding Pandemics). A fourth stage, representing
the Age of Delayed Aging has been suggested by Olshansky and Ault (1986),
thereby identifying the decline in CVD mortality from the 1960s onwards.  

The theory of rectangularisation of mortality is connected to Fries
(1980). He argued that there are biological limits to the human life
span. Even though we eliminate premature and avoidable deaths, senescence
will eventually bring an end to life. He suggested that 85 years would
represent an end in increasing life expectancies. Others, on the other hand
have argued that we still do not see any evidence of a slowing down of
increasing life expectancies, and that we have no proof of an end or that
it is far away. 
 
As Wilmoth and Horiuchi (1999) remarks, there is no necessary association
between rectangularization and a fixed life span. The opposite, that if
there is a fixed maximum age, then rectangularization is a necessary
consequence. 

Present health conditions among the elderly is not dependent on the
conditions in the world they are living in right now. Previous experiences
are important and a life course approach therefore important (Kuh and Ben
Schlomo for example). Different kinds of hazards and insults during life
have effects later in life. Early life experiences are an important issue
here, as demonstrated by several researchers. There has been several
pathways suggested for this, for example the effect of inflammations (Finch
and Crimmins).

\subsection{Method}

As discussed above, different methods have been used for the studies of
distribution of death age. If the focus is on the maximal life span, then
the analysis ought to be performed only on the oldest age groups since the
results would otherwise be strongly influenced by mortality in young ages
that would not be informative for such questions (Kannisto). One method
developed for this purpose is to include only mortality above modal age and
to analyze the dispersion of death age for that group and it this method we
start with. Alternative methods are also possible. For the complete period
(1900-2014), we use different measures on distribution of death age.   

We investigate the development according to the large cause-of-death
categories. We believe that this will improve the explanatory value of the
research since different cause-of-death categories have very different
distribution of death ages. This study is possible to perform on data for
the period 1952-2010 where we have access to exact age (in years) at death
according to cause of death for all Swedes dying during these years. 

Education, income, residence as explanations from the Linnaeus database.

We will also look at old age mortality from a cohort perspective. (?)  

 
\subsection{Sources}

The main source for the project is the Linnaeus database at CEDAR and DDB,
Umeå University. It contains all deaths with causes of death from 1961
onwards as well as the complete population from the census every fifth year
1960 to 1990 as well as the yearly population from the LISA database at
Statistics Sweden 1986-2014. Individuals in the Linnaeus database are
linked, which makes it possible to analyze the cause-specific mortality
from age, family relations, residence, education and socio-economic
position.  

\textcolor{red}{
(Other and alternative sources: Cause of death statistics from National
Board of Health and Welfare (Socialstyrelsen) in digital form can be
stretched back to 1952. }

\textcolor{red}{The distribution of all-cause mortality is possible to study from the early
20th century by adding data for the period before 1952 from Sveriges
dödbok. The data for the older period has been entered by volunteers from
Sveriges släktforskarförbund and is now complete.} 

\textcolor{red}{A possible additional source for the extended project is the Poplink
database at DDB. It allows us to study cause-specific rectangularization in
local environments (the Skellefteå and Umeå region) from 1910 to the 1960s
and to analyse the questions from a cohort perspective.
}

\textcolor{red}{Other possible sources: The Dödsorsaker-database with national and regional
age- and sex-specific causes of death 1911-2000 (?) stored at the
department of epidemiology.)}

\textcolor{red}{Human Mortality.org, a database of age-specific death rates from oldest
available time. Both period and cohort data.)}


\subsection{Time plan}

Start 2017, and continuing to the end of 2018 (?).

\subsection{Genomförande}

The basic research questions are the following:

\begin{itemize}

\item[] Do we find a rectangularization of mortality in Sweden 1900-2014,
  i.e. is age at death compressed towards the highest ages?

\item[] Are there different patterns of rectangularization according to
  cause-of-death category in Sweden 1952--2014?

\item[] To what extent is a possible rectangularization related to changes
in population structure or changing patterns of mortality inequality within
the population when it comes to education, residence and socio-economic
status?  

\item[] (Do we find cohort effects of old age mortality in relation to
  changes, for example the improved control of and lower mortality from
  infectious diseases from the period after the First World War, and the
  introduction of improved medicine from the 1940s onwards?) 

\item[] (Simulations and projections into the future?) 

\end{itemize}

\subsection{Project organisation}
		
\begin{itemize}

\item[] Sören Edvinsson 	
\item[]Erling Häggström
\item[]Göran Broström	
\end{itemize}

\section{Significance of the project and added value to the research field}

One of the major challenges in Sweden as well as in other countries around
the world is the aging population. The aging population has been driven
both by declining fertility as well as increased survival during the last
century, and we have no reasons to believe that fertility will change
radically and thereby ameliorate the aging process. The development of the
aging population is strongly determined by the future development of life
expectancy and mortality in old age. Life expectancy has continued to
increase in a surprisingly stable way – in Sweden a couple of months per
calendar year for several decades now. A better understanding on the coming
development of mortality will enable us to project the national population
development as well as changes in age structure. Demographic development
represents long-term processes where previous and present circumstances
have implications long into the future. It is therefore necessary to have a
clear picture of these processes, i.e. have a historical perspective on the
development.
 
Compression of mortality is a different theme from that of compression of
morbidity. It does not in itself say anything about a possible shortening
of time of bad health towards the end of life. Rectangularisation do
however have implications for the theory of morbidity compression. Fries
argues that a development towards a later onset of disability will
necessarily lead to a shorter period of such a condition if there really is
a fixed maximum life span.
  
The development of population size in general and the age structure of the
population in particular (the aging population). If we continue to increase
survival at the same pace as now, the population will increase rapidly. A
larger part of the population will be in age group above 65. This has
implications for retirement systems, and working life. It may also be
relevant for health care needs and other needs.

It is still unclear what this will mean for the development of the economy,
what the future costs will be and if health care and other costs will
increase. 

Updating the development to recent time is valuable, because the levels of
“premature” deaths are now very low. 

\section{Preliminary results}

Some initial analysis of rectangularization of mortality (measured by
distribution of death age above modal age of death) for the period
1952-2010 indicates no compression for most of the period, but signs of
compression from the 1990’s onwards (although rather weak) thereafter. When
separating the analysis between the major groups of cause of death
(cardiovascular, cancers, digestive and respiratory), we find a similar
indication of decreasing variation from around the 1990’s but with somewhat
different patterns depending on group. 

\section{Need for infrastructure}

Linnaeus database.

\section{International collaboration}

\section{References}


\end{document}

